\chapter{Introduction}
%%%%%%%%%%%%%%%%%%%%%%%%%%%%%%%%%%%%%%%%%%%%%%%%%%%%%%
% Comment: Just start directly with the motivation and the objectives of your thesis (without any further sections and subsections (1-2 pages)
%%%%%%%%%%%%%%%%%%%%%%%%%%%%%%%%%%%%%%%%%%%%%%%%%%%%%%
The introduction is a short overview of the context and goals of the paper.
It describes the structure and contents of the paper
to provide the reader with enough information
to find parts of relevance to their current interest quickly.

If you are unfamiliar with {\LaTeX},
you will probably need to take some time to learn the general workflow.
However, this will save you a lot of time and headaches later on,
as it allows you to completely focus on the content
instead of managing layouts and correct citation.
There are multiple great guides and introductions on using it.
\footnote{Overleaf: \url{https://www.overleaf.com/learn/latex/Tutorials}}
\footnote{Tutorial: \url{http://alexanderfabisch.github.io/latex-for-dissertations.html}}
%%%%%%%%%%%%%%%%%%%%%%%%%%%%%%%%%%%%%%%%%%%%%%%%%%%
\newpage
\textbf{Example of introduction:}\newline
The brain, spinal cord, and nerves build the nervous system. They control all the workings of the body. When there is a problem in this system that causes some troubles in regular activities of humans such as moving, speaking, and learning, it can also cause some problems in your memory, sense, or even mood.
Degenerative diseases, where nerve cells are damaged or die, like Parkinson's disease and Alzheimer's disease or retinitis pigmentosa, which causes severe long-term visual impairment in the retina, where the photosensitive layer in the retina dies, are some examples of neurological diseases.
Therefore research and implementation methods for solutions or treatments for these disorders can be one of the most exciting areas in both medical and engineering fields~\cite{Lee2021}. However, real-time and direct monitoring of the brain condition, accurate sensing of the brain activity, and having feedback to control the application for therapies due to the poor accessibility to the brain because of the brain's structure can be challenging~\cite{Lee2021}.

This thesis consists of two different parts which are hardware and software. In the hardware part, we have developed an Embedded setup with converting digital data from the neural datasets to the electrical analog signal for testing our data from software parts and also demonstrating the data which will be applied to~\ac{AFE} in future implants which are implemented in~\ac{ASIC}. 
In the Software part, a Python environment should be developed in order (a) to consider different methods of data compression for an event-based sampling of extracellular recordings and (b) to optimize the pipeline of the~\ac{AI}-enhanced spike sorting and detection with~\ac{ML} techniques. 
